\htmlhr
\chapterAndLabel{Linear Checker}{linear-checker}

The Linear Checker implements type-checking for a linear type system.  A
linear type system prevents re-usability:  there is only one (usable) reference
to a given object at any time.  Once a reference appears on the right-hand
side of an assignment in the form of a method invocation, it may not be used any more.  The same rule applies
for pseudo-assignments such as procedure argument-passing (including as the
receiver) or return.

One way of thinking about this is that a reference can only be used once,
after which it is ``used up''.  This property is checked statically at
compile time.  The single-use property only applies to use in an
assignment, which makes a new reference to the object; ordinary field
de-referencing does not use up a reference.

By forbidding re-usability, a linear type system can prevent problems for variables that cannot be used more than once.
The java.util.Stream library is one example. Streams cannot be operated on more than once, hence, they cause the program
to fail when they are used more than one time. By annotating these streams as @Linear, these errors can be observed at
compile time.

To run the Linear Checker, supply the
\code{-processor org.checkerframework.checker.linear.LinearChecker}
command-line option to javac.

Figure~\ref{fig-linear-example} gives an example of the Linear Checker's rules.

\begin{figure}
%BEGIN LATEX
\begin{smaller}
%END LATEX
\begin{Verbatim}
public static void main() {
    @Linear String a = "hi";
    String b = a.toUpperCase();   // Since 'a' is operated upon (by method invocation), 'a' has now become @Unusable
    String c = a.toLowerCase();  // ERROR : 'a' is @Unusable and can no longer be used again
}
\end{Verbatim}
%BEGIN LATEX
\end{smaller}
%END LATEX
\caption{Example of Linear Checker rules.}
\label{fig-linear-example}
\end{figure}

\sectionAndLabel{Linear annotations}{linear-annotations}

The full qualifier hierarchy for the linear type system includes three
types:

\begin{itemize}
\item
\code{@Normal} annotation means that the annotated objects can be assigned or used any number of times.  This is the default, so users only
need to write \code{@Normal} if they change the default.
\item
\code{@Unusable} annotation means that the annotated objects can only be operated upon once. After this operation, they become @Unusable.
@Linear objects can be assigned or referenced any number of times, but can accommodate only one method invocation.
\item
\code{@Unusable} annotation means that the annotated objects can not be either assigned, referenced or operated upon again.
Doing so would invoke the [use.unsafe] error in the compiler.
\end{itemize}

\noindent
\code{@Normal} is a supertype of \code{@Linear}, which is a
supertype of \code{@Unusable}.
