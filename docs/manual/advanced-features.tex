\htmlhr
\chapterAndLabel{Advanced type system features}{advanced-type-system-features}

This chapter describes features that are automatically supported by every
checker written with the Checker Framework.
You may wish to skim or skip this chapter on first reading.  After you have
used a checker for a little while and want to be able to express more
sophisticated and useful types, or to understand more about how the Checker
Framework works, you can return to it.


\sectionAndLabel{Invariant array types}{invariant-arrays}

Java's type system is unsound with respect to arrays.  That is, the Java
type-checker approves code that is unsafe and will cause a run-time crash.
Technically, the problem is that Java has ``covariant array types'', such
as treating \<String[]> as a subtype of \<Object[]>.  Consider the
following example:

\begin{Verbatim}
  String[] strings = new String[] {"hello"};
  Object[] objects = strings;
  objects[0] = new Object();
  String myString = strs[0];
\end{Verbatim}

\noindent
The above code puts an \<Object> in the array \<strings> and thence in
\<myString>, even though \<myString = new Object()> should be, and is,
rejected by the Java type system.  Java prevents corruption of the JVM by
doing a costly run-time check at every array assignment; nonetheless, it is
undesirable to learn about a type error only via a run-time crash rather
than at compile time.

When you pass the \<-AinvariantArrays> command-line option,
the Checker Framework is stricter than Java, in the sense that it treats
arrays invariantly rather than covariantly.  This means that a type system
built upon the Checker Framework is sound:  you get a compile-time
guarantee without the need for any run-time checks.  But it also means that
the Checker Framework rejects code that is similar to what Java unsoundly
accepts.  The guarantee and the compile-time checks are about your
extended type system.  The Checker Framework does not reject the example
code above, which contains no type annotations.

Java's covariant array typing is sound if the array is used in a read-only
fashion:  that is, if the array's elements are accessed but the array is
not modified.  However, facts about read-only usage are not built into any of
the type-checkers.  Therefore, when using type systems
along with \<-AinvariantArrays>, you will need to suppress any warnings that
are false positives because the array is treated in a read-only way.


\sectionAndLabel{Context-sensitive type inference for array constructors}{array-context-sensitive}

When you write an expression, the Checker Framework gives it the most
precise possible type, depending on the particular expression or value.
For example, when using the Regex Checker (\chapterpageref{regex-checker}),
the string \<"hello"> is given type \<@Regex String> because it is a legal
regular expression (whether it is meant to be used as one or not) and the
string \<"(foo"> is given the type \<@RegexBottom String> because it is not
a legal regular expression.

Array constructors work differently.  When you create an array with the
array constructor syntax, such as the right-hand side of this assignment:

\begin{Verbatim}
String[] myStrings = {"hello"};
\end{Verbatim}

\noindent
then the expression does not get the most precise possible type, because
doing so could cause inconvenience.  Rather, its type is determined by the
context in which it is used:  the left-hand side if it is in an assignment,
the declared formal parameter type if it is in a method call, etc.

In particular, if the expression \verb|{"hello"}| were given the type
\<@Regex String[]>, then the assignment would be illegal!  But the Checker
Framework gives the type \<String[]> based on the assignment context, so the code
type-checks.

If you prefer a specific type for a constructed array, you can indicate
that either in the context (change the declaration of \<myStrings>) or in a
\<new> construct (change the expression to \<new @Regex String[] >\verb|{"hello"}|).


\sectionAndLabel{Upper bound of qualifiers on uses of a given type}{upper-bound-for-use}
The examples in this section use the type qualifier hierarchy \code{@A :> @B :> @C}.

A qualifier on a use of a certain type must be a subtype or equal to the upper bound for that type.
The upper bound of qualifiers used on a given type is specified by annotating the type declaration
with some qualifier --- that is, by writing an annotation on a class declaration.
\begin{Verbatim}
  @C class MyClass {}
\end{Verbatim}

This means that \<@B MyClass> is an invalid type.  (Annotations on class declarations may also specify
default annotations for uses of the type; see Section~\ref{default-for-use})

If it is not possible to annotate the class's definition (e.g., for
primitives and some library classes),
the type-system designer can specify an upper bound by using the meta-annotation
\refqualclass{framework/qual}{UpperBoundFor}.

If no annotation is present on a type declaration and if no \<@UpperBoundFor> mentions the type, then
the bound is top. This can be changed by overriding
\refmethodanchortext{framework/type}{AnnotatedTypeFactory}{getTypeDeclarationBounds}{-javax.lang.model.type.TypeMirror-}{AnnotatedTypeFactory\#getTypeDeclarationBounds}.

There are two exceptions.
\begin{itemize}
  \item
  An expression can have a supertype of the upper bound; that is, some expression could
  have type \<@B MyClass>.  This type is not written explicitly, but results from viewpoint adaptation.
  \item
  Using usual CLIMB-to-top rules, local variables of type MyClass default to \<@A MyClass>.
  It is legal for \<@A MyClass> to be the type of a local variable.
  For consistency, users are allowed to write such a type on a local variable declaration.
\end{itemize}

Due to existing type rules, an expression of type \<@A MyClass> can only be used in limited ways.
\begin{itemize}
  \item
  Since every field, formal parameter, and return type of type MyClass (or lower) is annotated as
  \<@B> (or lower), it cannot be assigned to a field, passed to a method, or returned from a method.
  \item
  It can be used in a context that requires \<@A Object> (or whatever the least supertype is of MyClass
  for which the \<@A> qualifier is permitted).  Examples include being tested against \<null> or
  (for most type systems) being passed to polymorphic routines such as \<System.out.println> or \<System.identityHashCode>.
\end{itemize}

These operations might refine its type.  If a user wishes to annotate a method that does type refinement,
its formal parameter must be of illegal type \<@A MyClass>, which requires a warning suppression.

If the framework were to forbid expressions and local variables from having types inconsistent with the class annotation,
then important APIs and common coding paradigms would no longer type-check.

Consider the annotation
\begin{Verbatim}
  @NonNull class Optional { ... }
\end{Verbatim}
and the client code

\begin{Verbatim}
  Map<String, Optional> m;
  String key = ...;
  Optional value = m.get(key);
  if (value != null) {
  ...;
  }
\end{Verbatim}

The type of \<m.get(key)> is \<@Nullable Optional>, which is an illegal type.
However, this is a very common paradigm.  Programmers should not need to rewrite the code to test
\<m.containsKey(key)> nor suppress a warning in this safe code.

\sectionAndLabel{The effective qualifier on a type (defaults and inference)}{effective-qualifier}

A checker sometimes treats a type as having a slightly different qualifier
than what is written on the type --- especially if the programmer wrote no
qualifier at all.
Most readers can skip this section on first reading, because you will
probably find the system simply ``does what you mean'', without forcing
you to write too many qualifiers in your program.
In particular, qualifiers in method bodies are rare (except occasionally on
type arguments and array component types).

The following steps determine the effective
qualifier on a type --- the qualifier that the checkers treat as being present.

\begin{enumerate}
\item
  If a type qualifier is present in the source code, that qualifier is used.

\item
  If there is no explicit qualifier on a type, then a default
  qualifier
  % except for type parameters, but don't clutter this text with that detail
  is applied; see Section~\ref{defaults}. Defaulted qualifiers are treated by checkers
  exactly as if the programmer had written them explicitly.

\item
  The type system may refine a qualified type on a local variable --- that
  is, treat it as a subtype of how it was declared or defaulted.  This
  refinement is always sound and has the effect of eliminating false
  positive error messages.  See Section~\ref{type-refinement}.

\end{enumerate}

%TODO: Where does @QualifierForLiterals go?

\sectionAndLabel{Default qualifier for unannotated types}{defaults}

An unannotated
Java type is treated as if it had a default annotation.
Both the type system designer and an end-user programmer can control the defaulting.
Defaulting never applies to uses of type variables, even if they do not
have an explicit type annotation.
% TODO: If the type of a local variable is a use of a type parameter, then it is defaulted to top,
% so it can be refined. I don't know that's worth mentioning here.  Maybe just change never to does not?
Most of this section is about defaults for source code that is read by
the compiler.  When the compiler reads a \<.class> file, different
defaulting rules apply.
See Section~\ref{defaults-classfile} for these rules.

There are several defaulting mechanisms, for convenience and flexibility.
When determining the default qualifier for a use of an unannotated type, \<MyClass>, the following
rules are used in order, until one applies.
%TODO: I don't think this is true.(Some of these currently do not work in stub files.)

\begin{enumerate}
\item
  The qualifier specified via \<@DefaultQualifierForUse> on the declaration of the \<MyClass>.
  (Section~\ref{default-for-use})
\item
  If no \<@NoDefaultQualifierForUse> is written on the declaration of \<MyClass>, the qualifier
  explicitly written on the declaration of \<MyClass>. (Section~\ref{default-for-use})
\item
  The qualifier with a meta-annotation \<@DefaultFor(types = MyClass.class)>. (Section~\ref{default-for-use})
\item
  The qualifier with a meta-annotation \<@DefaultFor(typeKinds = KIND)>, where \<KIND> is the
  \<TypeKind> of \<MyClass>. (Section~\ref{default-for-use})
\item
  The qualifier in the innermost user-written \<@DefaultQualifier> for the location of the use of \<MyClass>.
  (Section~\ref{default-qualifier})
\item
  The qualifier in the meta-annotation \<@DefaultFor> for the location of the use of \<MyClass>.
  These are defaults specified by the type system designer (Section~\ref{creating-typesystem-defaults});
  this is usually CLIMB-to-top (Section~\ref{climb-to-top}) (CLIMB means \textbf{C}asts, \textbf{L}ocals,
  \textbf{I}nstanceof, and (some) i\textbf{M}plicit \textbf{B}ounds).
\item
  The qualifier with the meta-annotation \refqualclass{framework/qual}{DefaultQualifierInHierarchy}.
\end{enumerate}

If the unannotated type is the type of a local variable, then the first 5 rules are skipped and only
rules 6 and 7 apply. If rule 6 applies, it makes the type of local variables top so they can be refined.

% (Implementation detail:  setting defaults is implemented by the
% \refclass{framework/util}{QualifierDefaults} class.)

\subsectionAndLabel{Default for use of a type}{default-for-use}
The type declaration annotation \refqualclass{framework/qual}{DefaultQualifierForUse}
indicates that the specified qualifier should be added to all unannotated uses of the type.

For example:
\begin{Verbatim}
@DefaultQualifierForUse(B.class)
class MyClass {}
\end{Verbatim}

This means any unannotated use of \<MyClass> is treated as \<@B MyClass> by the checker.
(Except for locals, which can be refined.)

Similarly, the meta-annotation \refqualclass{framework/qual}{DefaultFor} can be used to specify defaults
for uses of of types, using the \<types> element, or type kinds, using the \<typeKinds> elements.

Interaction between qualifier bounds and \<DefaultQualifierForUse>:
\begin{itemize}
\item
  If a type declaration is annotated with a qualifier bound, but not a \<@DefaultQualifierForUse>,
  then the qualifier bound is added to all unannotated uses of that type (except locals).
  For example, \<@C class MyClass {}> is equivalent to
\begin{Verbatim}
@DefaultQualifierForUse(C.class)
@C class MyClass {}
\end{Verbatim}

\item
  If the qualifier bound should not be added to all unannotated uses, then
  \refqualclass{framework/qual}{NoDefaultQualifierForUse} should be written on the declaration:
\begin{Verbatim}
@NoDefaultQualifierForUse
@C class MyClass {}
\end{Verbatim}
  This means that unannotated uses of MyClass are defaulted normally.
\item
  If neither \<@DefaultQualifierForUse> nor a qualifier bound is present on a type declaration, that
  is equivalent to writing \<@NoDefaultQualifierForUse>.

\end{itemize}

\subsectionAndLabel{Controlling defaults in source code}{default-qualifier}
The end-user programmer specifies a default qualifier by writing the
\refqualclass{framework/qual}{DefaultQualifier}\code{(\emph{ClassName}, }[\emph{locations}]\<)>
annotation on a package, class, method, or variable declaration.  The
argument to \refqualclass{framework/qual}{DefaultQualifier} is the \code{Class}
name of an annotation.
The optional second argument indicates where the default
applies.  If the second argument is omitted, the specified annotation is
the default in all locations.  See the Javadoc of \refclass{framework/qual}{DefaultQualifier} for details.

For example, using the Nullness type system (Chapter~\ref{nullness-checker}):

\begin{Verbatim}
import org.checkerframework.framework.qual.DefaultQualifier;
import org.checkerframework.checker.nullness.qual.NonNull;

@DefaultQualifier(NonNull.class)
class MyClass {

  public boolean compile(File myFile) { // myFile has type "@NonNull File"
    if (!myFile.exists())          // no warning: myFile is non-null
      ...
    @Nullable File srcPath = ...;  // must annotate to specify "@Nullable File"
    if (srcPath.exists())          // warning: srcPath might be null
      ...
  }

  @DefaultQualifier(Tainted.class)
  public boolean isJavaFile(File myfile) {  // myFile has type "@Tainted File"
    ...
  }
}
\end{Verbatim}

You may write multiple
\refqualclass{framework/qual}{DefaultQualifier} annotations at a single location.

If \code{@DefaultQualifier}[\code{s}] is placed on a package (via the
\<package-info.java> file), then it applies to the given package \emph{and}
all subpackages.
% This is slightly at odds with Java's treatment of packages of different
% names as essentially unrelated, but is more intuitive and useful.


%% Don't even bother to bring this up; it will just sow confusion without
%% being helpful.
% For some type systems, a user may not specify a default qualifier, or doing
% so prevents giving any other qualifier to any reference.  This is a
% consequence of the design of the type system; see
% Section~\ref{bottom-and-top-qualifier}.


%When a programmer omits an \<extends> clause at a declaration of a type
%parameter, the implicit upper bound is defaulted to the top qualifier; see
%Section~\ref{climb-to-top}.


\subsectionAndLabel{Defaulting rules and CLIMB-to-top}{climb-to-top}

Each type system defines a default qualifier (see
Section~\ref{creating-typesystem-defaults}).  For example, the default
qualifier for the Nullness Checker is
\refqualclass{checker/nullness/qual}{NonNull}.  When a user
writes an unqualified type such as \<Date>, the Nullness Checker interprets it as
\<@NonNull Date>.

The type system applies that default qualifier to most but
not all type uses.  In particular, unless otherwise stated, every type system
uses the CLIMB-to-top rule.  This
rule states that the \emph{top} qualifier in the hierarchy is the default for
the CLIMB locations:  \textbf{C}asts, \textbf{L}ocals, \textbf{I}nstanceof,
and (some) i\textbf{M}plicit \textbf{B}ounds.
For example, when the user writes an unqualified type such as \<Date> in such a
location, the Nullness Checker interprets it as \<@Nullable Date> (because
\refqualclass{checker/nullness/qual}{Nullable} is the top qualifier in the
hierarchy, see Figure~\ref{fig-nullness-hierarchy}).

% TODO:  Add wildcard bounds to the set of type-refined locations?
% Especially in light of
% https://github.com/typetools/checker-framework/issues/260

The CLIMB-to-top rule is used only for unannotated source code that is
being processed by a checker.  For unannotated libraries (code read by the
compiler in \<.class> or \<.jar> form), see Section~\ref{defaults-classfile}.

The rest of this section explains the rationale and implementation of
CLIMB-to-top.

Here is the rationale for CLIMB-to-top:

\begin{itemize}
\item
Local variables are defaulted to top because type refinement
(Section~\ref{type-refinement}) is applied to local variables.  If a local
variable starts as the top type, then the Checker Framework refines it to
the best (most specific) possible type based on assignments to it.  As a
result, a programmer rarely writes an explicit annotation on any of those
locations.

Variables defaulted to top include local variables, resource variables in the
try-with-resources construct, variables in \<for> statements, and \<catch>
arguments (known as exception parameters in the Java Language Specification).
Exception parameters need to have the top type because
exceptions of arbitrary qualified types can be thrown and the Checker Framework
does not provide run-time checks.

\item
Cast and instanceof types
are given the same type as their argument expression.  This has the same
effect as if they were given the
top type and then flow-sensitively refined to the type of their argument.
However, note that programmer-written type qualifiers are \emph{not}
refined.
% Programmer-written type qualifiers could be refined, but it has never
% been a priority.  The code that defaults casts in this manner predates
% the dataflow framework and we never reimplemented casts in the dataflow
% framework.

\item
Implicit upper bounds are defaulted to top to allow them to be instantiated
in any way.  If a user declared \code{class C<T> \ttlcb\ ...\ \ttrcb}, then
the Checker Framework assumes that the user intended to allow any instantiation of the class,
and the declaration is interpreted as \code{class C<T extends @Nullable
  Object> \ttlcb\ ...\ \ttrcb} rather than as \code{class C<T extends
  @NonNull Object> \ttlcb\ ...\ \ttrcb}.  The latter would forbid
instantiations such as \code{C<@Nullable String>}, or would require
rewriting of code.  On the other hand, if a user writes an explicit bound
such as \code{class C<T extends D> \ttlcb\ ...\ \ttrcb}, then the user
intends some restriction on instantiation and can write a qualifier on the
upper bound as desired.

This rule means that the upper bound of \code{class C<T>} is defaulted
differently than the upper bound of \code{class C<T extends Object>}.
This may seem confusing, but it is the least bad option.  The
more confusing alternative would be for ``\code{Object}'' to be defaulted
differently in \code{class C<T extends Object>} and in an
instantiation \code{C<Object>}, and for the upper bounds to be defaulted
differently in \code{class C<T extends Object>}
and \code{class C<T extends Date>}.

\item
Implicit \emph{lower} bounds are defaulted to the bottom type, again to allow
maximal instantiation.  Note that Java does not allow a programmer to
express both the upper and lower bounds of a type, but the Checker
Framework allows the programmer to specify either or both;
see Section~\ref{generics-defaults}.

\end{itemize}

A \refqualclass{framework/qual}{DefaultQualifier} that specifies a
CLIMB-to-top location takes precedence over the CLIMB-to-top rule.

Here is how the Nullness Checker overrides part of the CLIMB-to-top rule:

\begin{Verbatim}
@DefaultQualifierInHierarchy
@DefaultFor({ TypeUseLocation.EXCEPTION_PARAMETER })
public @interface NonNull {}

public @interface Nullable {}
\end{Verbatim}

\noindent
 As mentioned above, the exception parameters are always non-null, so
\<@DefaultFor(\{ TypeUseLocation.EXCEPTION\_PARAMETER \})> on \<@NonNull> overrides
the CLIMB-to-top rule.


\subsectionAndLabel{Inherited defaults}{inherited-defaults}

In certain situations, it would be convenient for an annotation on a
superclass member to be automatically inherited by subclasses that override
it.  Such a feature would reduce annotation effort, but it would also reduce program
comprehensibility.  In general, a program is read more often than it is
edited/annotated, so the Checker Framework does not currently support this
feature.

Currently, a user can determine the annotation on a parameter or return
value by looking at a single file.  If annotations could be inherited from
supertypes, then a user would have to examine all supertypes, and do
computations over them, to understand the meaning of an unannotated type in
a given file.

Computation is necessary because different annotations might be inherited
from a supertype and an interface, or from two interfaces.  For return
types, the inherited type should be the least upper bound of all
annotations on overridden implementations in supertypes.  For method
parameters, the inherited type should be the greatest lower bound of all
annotations on overridden implementations in supertypes.  In each case, an
error would be thrown if no such annotations existed.

In the future, this feature may be added optionally, and each type-checker
implementation can enable it if desired.


\subsectionAndLabel{Inherited wildcard annotations}{inherited-wildcard-annotations}

If a wildcard is unbounded and has no annotation (e.g. \code{List<?>}),
the annotations on the wildcard's bounds are copied from the type parameter
to which the wildcard is an argument.

For example, the two wildcards in
the declarations below are equivalent.

\begin{Verbatim}
class MyList<@Nullable T extends @Nullable Object> {}

MyList<?> listOfNullables;
MyList<@Nullable ? extends @Nullable Object> listOfNullables;
\end{Verbatim}

The Checker Framework copies
these annotations because wildcards must be within the bounds of their
corresponding type parameter.
By contrast, if the bounds of a wildcard
were defaulted differently from the bounds of its corresponding type
parameter, then there would be many false positive
\code{type.argument.type.incompatible} warnings.

Here is another example of two equivalent wildcard declarations:

\begin{Verbatim}
class MyList<@Regex(5) T extends @Regex(1) Object> {}

MyList<?> listOfRegexes;
MyList<@Regex(5) ? extends @Regex(1) Object> listOfRegexes;
\end{Verbatim}

Note, this copying of annotations for a wildcard's bounds applies only to
unbounded wildcards.  The two wildcards in the
following example are equivalent.

\begin{Verbatim}
class MyList<@NonNull T extends @Nullable Object> {}

MyList<? extends Object> listOfNonNulls;
MyList<@NonNull ? extends @NonNull Object> listOfNonNulls2;
\end{Verbatim}

Note, the upper bound of the wildcard \code{?~extends Object} is defaulted to
\code{@NonNull} using the CLIMB-to-top rule (see Section~\ref{climb-to-top}).
Also note that the \code{MyList} class declaration could have been more succinctly
written as: \code{class MyList<T extends @Nullable Object>} where the lower bound
is implicitly the bottom annotation: \code {@NonNull}.

\subsectionAndLabel{Default qualifiers for \<.class> files (library defaults)}{defaults-classfile}

(\emph{Note:} Currently, the conservative library defaults presented in this section
are off by default and can be turned on by supplying the \<-AuseConservativeDefaultsForUncheckedCode=bytecode>
command-line option.  In a future release, they will be turned on
by default and it will be possible to turn them off by supplying a
\<-AuseConservativeDefaultsForUncheckedCode=-bytecode> command-line option.)

The defaulting rules presented so far apply to source code that is read by
the compiler.  When the compiler reads a \<.class> file, different
defaulting rules apply.

If the checker was run during the compiler execution that created the
\<.class> file,
%% This caveat is true, but it's a distraction at this point in the manual.
%% Also, not having a top and a bottom qualifier is an uncommon case.
% (and the qualifier hierarchy has both a top and a bottom
% qualifier, see Section~\ref{bottom-and-top-qualifier}),
then there is no need for
defaults:  the \<.class> file has an explicit qualifier at each type use.
(Furthermore, unless warnings were suppressed, those qualifiers are
guaranteed to be correct.)
When you are performing pluggable type-checking,
it is best to ensure that the compiler only reads such \<.class> files.
Section~\ref{compiling-libraries} discusses how to create annotated
libraries.
%% True, but not relevant to the point of this paragraph.
% , even if the library source code is only partially annotated.

If the checker was not run during the compiler execution that created the
\<.class> file, then the \<.class> file contains only the type qualifiers
that the programmer wrote explicitly.  (Furthermore, there is no guarantee
that these qualifiers are correct, since they have not been checked.)
In this case, each checker decides what qualifier to use for the
locations where the programmer did not write an annotation.  Unless otherwise noted, the
choice is:

\begin{itemize}
\item
  For method parameters and lower bounds, use the bottom qualifier (see
  Section~\ref{creating-bottom-qualifier}).
\item
  For method return values, fields, and upper bounds, use the top qualifier (see
  Section~\ref{creating-top-qualifier}).
\end{itemize}

These choices are conservative.  They are likely to cause many
false-positive type-checking errors, which will help you to know which
library methods need annotations.  You can then write those library
annotations (see Chapter~\ref{annotating-libraries}) or alternately
suppress the warnings (see Chapter~\ref{suppressing-warnings}).

For example, an unannotated method

\begin{Verbatim}
  String concatenate(String p1, String p2)
\end{Verbatim}

\noindent
in a classfile would be interpreted as

\begin{Verbatim}
  @Top String concatenate(@Bottom String p1, @Bottom String p2)
\end{Verbatim}

There is no single possible default that is sound for fields.  In the rare
circumstance that there is a mutable public field in an unannotated
library, the Checker Framework may fail to warn about code that can
misbehave at run time.

%% TODO: The following rule is preferable to the current safe behavior.
%% However, we have not yet figured out how to implement separate handling
%% for field read vs. write.
% For fields, use the bottom qualifier when writing to the field and the
% top qualifier when reading from the field.

%% TODO: should give rules for other locations, such as type parameters
%% (which should behave like fields), bounds, etc.

% If you supply the command-line option
% \<-AunsafeDefaultsForUnannotatedBytecode>,
% then the checker does defaulting for unannotated bytecode like it does for
% annotated source code.  In other words, when a type use in a \<.class> file
% has no explicit annotation, it is defaulted using the same rules as for the
% corresponding source code location.  You should only use this command-line
% option as temporary measure, because it is unsafe:  the checker might issue
% no warnings even though the code could violate the type guarantee at run
% time.  However, it can be useful when you are first annotating a codebase,
% to help you focus on errors within the codebase before you have annotated
% external libraries.


\newpage
\sectionAndLabel{Annotations on constructors}{annotations-on-constructors}

\subsectionAndLabel{Annotations on constructor declarations}{annotations-on-constructor-declarations}

An annotation on the ``return type'' of a constructor declaration indicates
what the constructor creates.  For example,

\begin{Verbatim}
@B class MyClass {
  @C MyClass() {}
}
\end{Verbatim}

\noindent
means that invoking that constructor creates a \<@C MyClass>.

The Checker Framework cannot verify that the constructor really creates
such an object, because the Checker Framework does not know the
type-system-specific semantics of the \<@C> annotation.
Therefore, if the constructor result type is different than the top annotation in the
hierarchy, the Checker Framework will issue a warning.
The programmer should check the annotation manually, then
suppress the warning.


\subsubsectionAndLabel{Defaults}{constructor-declaration-defaults}

If a constructor declaration is unannotated, it defaults to the same type as that of
its enclosing class (rather than the default qualifier in the hierarchy).
For example, the Tainting Checker (Chapter~\ref{tainting-checker}) has \code{@Tainted}
as its default qualifier. Consider the following class:

\begin{Verbatim}
  @Untainted class MyClass {
    MyClass() {}
  }
\end{Verbatim}

\noindent
The constructor declaration is equivalent to \<@Untainted MyClass() \ttlcb\ttrcb>.

The Checker Framework produces the same error messages for
explicitly-written and defaulted annotations.


\subsectionAndLabel{Annotations on constructor invocations}{annotations-on-constructor-invocations}

The type of a method call expression \<x.myMethod(y, z)> is determined by
the return type of the declaration of \<myMethod>.  There is no way to
write an annotation on the call to change its type.  However, it is
possible to write a cast:  \<(@Anno SomeType) x.myMethod(y, z)>.  The Checker
Framework will issue a warning that it cannot verify that the downcast is
correct.  The programmer should manually determine that the annotation is
correct and then suppress the warning.

A constructor invocation \<new MyClass()> is also a call, so its semantics
are similar.  The type of the expression is determined by the annotation on
the result type of the constructor declaration.  It is possible to write a cast
\<(@Anno MyClass) new MyClass()>.  The syntax \<new @Anno MyClass()> is shorthand
for the cast.  For either syntax, the Checker Framework will issue a
warning that it cannot verify that the cast is correct.  The programmer may
suppress the warning if the code is correct.


\sectionAndLabel{Type refinement (flow-sensitive type qualifier inference)}{type-refinement}

The
checkers treat expressions within a method body as having a
subtype of their declared or defaulted (Section~\ref{defaults})
type.

Type refinement eliminates some false positive warnings.
Type refinement also
reduces your burden of writing type qualifiers in your program,
since you do not need to write type qualifiers on local variables.


\subsectionAndLabel{Type refinement examples}{type-refinement-examples}

Here is an example for the Nullness Checker
(\chapterpageref{nullness-checker}).
\<myVar> is declared as \<@Nullable String>, but
it is treated as \<@NonNull String> within the body of the \<if> test.

\begin{Verbatim}
  @Nullable String myVar;
  ...                   // myVar has type @Nullable String here.
  myVar.hashCode();     // warning: possible dereference of null.
  ...
  if (myVar != null) {
    ...                 // myVar has type @NonNull String here.
    myVar.hashCode();   // no warning.
  }
\end{Verbatim}

Here is another example.
Note that the same expression may yield a
warning or not depending on its context (that is, depending on the current
type refinement).

\begin{Verbatim}
  @Nullable String myVar;
  ...                   // myVar has type @Nullable String
  myVar = "hello";
  ...                   // myVar has type @NonNull String
  myVar.hashCode();     // no warning
  ...
  myVar = myMap.get(someKey);
  ...                   // myVar has type @Nullable String
  myVar.hashCode();     // warning: posible dereference of null
\end{Verbatim}

Type refinement applies to every checker, including new
checkers that you write.  Here is an example for the Regex Checker
(\chapterpageref{regex-checker}):

\begin{Verbatim}
  void m2(@Unannotated String s) {
    s = RegexUtil.asRegex(s, 2);  // asRegex throws an exception if its argument is not
                                  // a regex with the given number of capturing groups
    ...   // s now has type "@Regex(2) String"
  }
\end{Verbatim}


\subsectionAndLabel{Type refinement behavior}{type-refinement-behavior}

The checker treats a variable or expression as a subtype of its declared type:
\begin{itemize}
\item
  starting at the time that
  it is assigned a value,
  a method establishes a postcondition (e.g., as
  expressed by \refqualclass{checker/nullness/qual}{EnsuresNonNull} or
  \refqualclass{framework/qual}{EnsuresQualifierIf}), or
  a run-time check is performed (e.g., via an assertion or
  \code{if} statement).
\item
until its value might change (e.g.,
via an assignment, or because a method call might have a side effect).
\end{itemize}

The checker never treats a variable as
a supertype of its declared type.  For example, an expression with declared type \refqualclass{checker/nullness/qual}{NonNull}
type is never treated as possibly-null, and such an assignment is always illegal.

The functionality has a variety of names:  automatic type refinement,
flow-sensitive type qualifier inference, local type inference, and
sometimes just ``flow''.


\subsectionAndLabel{Which types are refined}{type-refinement-which-types}

You generally do not need to annotate the top-level type of a local variable.
You do need to annotate its type arguments or array element types.
(Type refinement does not change them, because doing so would not produce a
subtype, as explained in see Section~\ref{covariant-type-parameters} and
Section~\ref{invariant-arrays}.)
Type refinement works within a method, so you still need to
annotate method signatures (parameter and return type) and field types.

If you find examples where you think a value should be inferred to have
(or not have) a
given annotation, but the checker does not do so, please submit a bug
report (see Section~\ref{reporting-bugs}) that includes a small piece of
Java code that reproduces the problem.


\subsubsectionAndLabel{Fields and type refinement}{type-refinement-fields}

Type refinement infers the type of fields in some restricted cases:

\begin{itemize}

\item
A final initialized field:
Type inference is performed for final fields that are initialized to a
compile-time constant at the declaration site; so the type of \code{protocol}
is \code{@NonNull String} in the following declaration:

\begin{Verbatim}
    public final String protocol = "https";
\end{Verbatim}

Such an inferred type may leak to the public interface of the class.
If you wish to override such behavior, you can explicitly insert the desired
annotation, e.g.,

\begin{Verbatim}
    public final @Nullable String protocol = "https";
\end{Verbatim}

\item
Within method bodies:
Type inference is performed for fields in the context of method bodies,
like local variables or any other expression.
Consider the following example, where \code{updatedAt} is a nullable
field:

\begin{Verbatim}
class DBObject {
  @Nullable Date updatedAt;

  void m() {
    // updatedAt is @Nullable, so warning about .getTime()
    ... updatedAt.getTime() ... // warning about possible NullPointerException

    if (updatedAt == null) {
      updatedAt = new Date();
    }

    // updatedAt is now @NonNull, so .getTime() call is OK
    ... updatedAt.getTime() ...
  }
}
\end{Verbatim}

A method call may invalidate inferences about field types; see
Section~\ref{type-refinement-purity}.

\end{itemize}


\subsectionAndLabel{Run-time tests and type refinement}{type-refinement-runtime-tests}

Some type systems support a run-time test that the Checker Framework can
use to refine types within the scope of a conditional such as \<if>, after
an \<assert> statement, etc.

Whether a type system supports such a run-time test depends on whether the
type system is computing properties of data itself, or properties of
provenance (the source of the data).  An example of a property about data is
whether a string is a regular expression.  An example of a property about
provenance is units of measure:  there is no way to look at the
representation of a number and determine whether it is intended to
represent kilometers or miles.

% Keep these lists in sync with the list in introduction.tex

Type systems that support a run-time test are:
\begin{itemize}
\item
  \ahrefloc{nullness-checker}{Nullness Checker} for null pointer errors
  (see \chapterpageref{nullness-checker})
\item
  \ahrefloc{map-key-checker}{Map Key Checker} to track which values are
  keys in a map (see \chapterpageref{map-key-checker})
\item
  \ahrefloc{optional-checker}{Optional Checker} for errors in using the
  \sunjavadoc{java.base/java/util/Optional.html}{Optional} type (see
  \chapterpageref{optional-checker})
\item
  \ahrefloc{lock-checker}{Lock Checker} for concurrency and lock errors
  (see \chapterpageref{lock-checker})
\item
  \ahrefloc{index-checker}{Index Checker} for array accesses
  (see \chapterpageref{index-checker})
\item
  \ahrefloc{regex-checker}{Regex Checker} to prevent use of syntactically
  invalid regular expressions (see \chapterpageref{regex-checker})
\item
  \ahrefloc{formatter-checker}{Format String Checker} to ensure that format
  strings have the right number and type of \<\%> directives (see
  \chapterpageref{formatter-checker})
\item
  \ahrefloc{i18n-formatter-checker}{Internationalization Format String Checker}
  to ensure that i18n format strings have the right number and type of
  \<\{\}> directives (see \chapterpageref{i18n-formatter-checker})
\end{itemize}


Type systems that do not currently support a run-time test, but could do so with some
additional implementation work, are

\begin{itemize}
\item
  \ahrefloc{interning-checker}{Interning Checker} for errors in equality
  testing and interning (see \chapterpageref{interning-checker})
\item
  \ahrefloc{propkey-checker}{Property File Checker} to ensure that valid
  keys are used for property files and resource bundles (see
  \chapterpageref{propkey-checker})
\item
  \ahrefloc{i18n-checker}{Internationalization Checker} to
  ensure that code is properly internationalized (see
  \chapterpageref{i18n-checker})
% The Compiler Message Key Checker is neither here nor in the introduction
% chapter because it is really meant for Checker Framework developers and
% as sample code, and is not meant for Checker Framework users at large.
\item
  \ahrefloc{signature-checker}{Signature String Checker} to ensure that the
  string representation of a type is properly used, for example in
  \<Class.forName> (see \chapterpageref{signature-checker}).
\item
  \ahrefloc{constant-value-checker}{Constant Value Checker} to determine
  whether an expression's value can be known at compile time
  (see \chapterpageref{constant-value-checker})
\end{itemize}


Type systems that cannot support a run-time test are:

\begin{itemize}
\item
  \ahrefloc{initialization-checker}{Initialization Checker} to ensure all
  fields are set in the constructor (see
  \chapterpageref{initialization-checker})
\item
  \ahrefloc{fenum-checker}{Fake Enum Checker} to allow type-safe fake enum
  patterns and type aliases or typedefs (see \chapterpageref{fenum-checker})
\item
  \ahrefloc{tainting-checker}{Tainting Checker} for trust and security errors
  (see \chapterpageref{tainting-checker})
\item
  \ahrefloc{guieffect-checker}{GUI Effect Checker} to ensure that non-GUI
  threads do not access the UI, which would crash the application
  (see \chapterpageref{guieffect-checker})
\item
  \ahrefloc{units-checker}{Units Checker} to ensure operations are
  performed on correct units of measurement
  (see \chapterpageref{units-checker})
\item
  \ahrefloc{signedness-checker}{Signedness Checker} to
  ensure unsigned and signed values are not mixed
  (see \chapterpageref{signedness-checker})
\item
  \ahrefloc{aliasing-checker}{Aliasing Checker} to identify whether
  expressions have aliases and prevent re-usage of objects (see \chapterpageref{aliasing-checker})
\item
  \ahrefloc{purity-checker}{Purity Checker} to identify whether
  methods have side effects (see \chapterpageref{aliasing-checker})
\item
  \ahrefloc{subtyping-checker}{Subtyping Checker} for customized checking without
  writing any code (see \chapterpageref{subtyping-checker})
\end{itemize}



\subsectionAndLabel{Side effects, determinism, purity, and type refinement}{type-refinement-purity}

Calling a method typically causes the checker to discard its knowledge of
the refined type, because the method might assign a field.
The \refqualclass{dataflow/qual}{SideEffectFree} annotation indicates that
the method has no side effects, so calling it does not invalidate any
dataflow facts.

Calling a method twice might have different results, so facts known about
one call cannot be relied upon at another call.
The \refqualclass{dataflow/qual}{Deterministic} annotation indicates that
the method returns the same result every time it is called on the same
arguments.

\refqualclass{dataflow/qual}{Pure} means both
\refqualclass{dataflow/qual}{SideEffectFree} and
\refqualclass{dataflow/qual}{Deterministic}.
The \refqualclass{dataflow/qual}{TerminatesExecution} annotation
indicates that a given method never returns.  This can enable the
type refinement to be more precise.

Chapter~\ref{purity-checker} gives more information about these annotations.
This section explains how to use them to improve type refinement.


\subsubsectionAndLabel{Side effects}{type-refinement-side-effects}

Consider the following declarations and uses:

\begin{Verbatim}
  @Nullable Object myField;

  int computeValue() { ... }

  void m() {
    ...
    if (myField != null) {
                        // The type of myField is now "@NonNull Object".
      int result = computeValue();
                        // The type of myField is now "@Nullable Object",
                        // because computeValue might have set myField to null.
      myField.toString(); // Warning: possible null pointer exception.
    }
  }
\end{Verbatim}

There are three ways to express that \<computeValue> does not set
\<myField> to \<null>, and thus to prevent the Nullness Checker from
issuing a warning about the call \<myField.toString()>.

\begin{enumerate}
\item
  If \<computeValue> has no side effects, declare it as

\begin{Verbatim}
  @SideEffectFree
  int computeValue() { ... }
\end{Verbatim}

\noindent
The Nullness Checker issues no warnings, because it can reason that
the second occurrence of \code{myField} has the same (non-null) value as
the one in the test.

\item
  If no method resets \<myField> to \<null> after it has been initialized
  to a non-null value (even if a method has some other side effect),
  declare \<myField> as \refqualclass{checker/nullness/qual}{MonotonicNonNull}.

\begin{Verbatim}
  @MonotonicNonNull Object myField;
\end{Verbatim}

\item
  If \<computeValue> sets \<myField> to a non-null value, declare it as

\begin{Verbatim}
  @EnsuresNonNull("myField")
  int computeValue() { ... }
\end{Verbatim}

  If \<computeValue> maintains \<myField> as a non-null value, even if it
  might have other side effects and even if other methods might set
  \<myField> to \<null>, declare it as

\begin{Verbatim}
  @RequiresNonNull("myField")
  @EnsuresNonNull("myField")
  int computeValue() { ... }
\end{Verbatim}
\end{enumerate}


\subsubsectionAndLabel{Deterministic methods}{type-refinement-determinism}

Consider the following declaration and uses:

\begin{Verbatim}
  @Nullable Object getField(Object arg) { ... }

  void m() {
    ...
    if (x.getField(y) != null) {
      x.getField(y).toString(); // warning: possible null pointer exception
    }
  }
\end{Verbatim}

The Nullness Checker issues a warning regarding the
\code{toString()} call, because its receiver \code{x.getField(y)} might
be \code{null}, according to the \code{@Nullable} return type in the
declaration of \code{getField}.  The Nullness Checker cannot assume that
\<getField> returns non-null on the second call, just based on the fact
that it returned non-null on the first call.

To indicate that a method returns the same value each time it is called on
the same arguments, use the \refqualclass{dataflow/qual}{Deterministic} annotation.
Actually, it is necessary to use \refqualclass{dataflow/qual}{Pure} which
means both \<@Deterministic> and \<@SideEffectFree>, because otherwise the
first call might change a value that the method depends on.

If you change the declaration of \code{getField} to

\begin{Verbatim}
  @Pure
  @Nullable Object getField(Object arg) { ... }
\end{Verbatim}

\noindent
then the Nullness Checker issues no warnings.
Because \<getField> is \<@SideEffectFree>, the values of \<x> and \<y> are the
same at both invocations.
Because \<getField> is \<@Deterministic>, the two invocations of
\code{x.getField(y)} have the same value.
Therefore, \code{x.getField(y)} is non-null within the \<then> branch
of the \<if> statement.



% This is a slightly funny place for this section, but I cannot decide on a
% better one.
\subsectionAndLabel{Assertions}{type-refinement-assertions}

If your code contains an \<assert> statement, then your code could behave
in two different ways at run time, depending on whether assertions are
enabled or disabled
via the \<-ea> or \<-da> command-line options to java.

By default, the Checker Framework outputs warnings about any error that
could happen at run time, whether assertions are enabled or disabled.

If you supply the \<-AassumeAssertionsAreEnabled> command-line option, then
the Checker Framework assumes assertions are enabled.  If you supply the
\<-AassumeAssertionsAreDisabled> command-line option, then the Checker
Framework assumes assertions are disabled.  You may not supply both
command-line options.  It is uncommon to supply either one.

These command-line arguments have no effect on processing of \<assert>
statements whose message contains the text \<@AssumeAssertion>; see
Section~\ref{assumeassertion}.


% If you add a Javadoc link to this location, also add the qualifier to the
% list below.
\sectionAndLabel{Writing Java expressions as annotation arguments}{java-expressions-as-arguments}

Sometimes, it is necessary to write a Java expression as the argument to an
annotation.  The annotations that take a Java
expression as an argument include:

% TODO: Need to periodically check/update this list.
\begin{itemize}
\item \refqualclass{framework/qual}{RequiresQualifier}
\item \refqualclass{framework/qual}{EnsuresQualifier}
\item \refqualclass{framework/qual}{EnsuresQualifierIf}
\item \refqualclass{checker/nullness/qual}{RequiresNonNull}
\item \refqualclass{checker/nullness/qual}{EnsuresNonNull}
\item \refqualclass{checker/nullness/qual}{EnsuresNonNullIf}
% Not implemented: \refqualclass{checker/nullness/qual}{AssertNonNullIfNonNull}
\item \refqualclass{checker/nullness/qual}{KeyFor}
\item \refqualclass{checker/nullness/qual}{EnsuresKeyFor}
\item \refqualclass{checker/nullness/qual}{EnsuresKeyForIf}
\item \refqualclass{checker/i18nformatter/qual}{I18nFormatFor}
\item \refqualclass{checker/lock/qual}{EnsuresLockHeld}
\item \refqualclass{checker/lock/qual}{EnsuresLockHeldIf}
\item \refqualclass{checker/lock/qual}{GuardedBy}
\item \refqualclass{checker/lock/qual}{Holding}
\end{itemize}

The set of permitted expressions is a subset of all Java expressions,
with a few extensions, formal parameters like \<\#1> and (for some type
systems) \code{<self>}.

\begin{itemize}
\item
  the receiver object, \<this>.  You can write \<this> to annotate any
  variable or declaration where you could write \<this> in code.
  Notably, it cannot be used in annotations on declarations of
  static fields or methods.  For a field, \<this> is the field's
  receiver, i.e. its container.  For a local variable, it is the
  method's receiver.

\item
  the receiver object as seen from the superclass, \<super>.  This can be used
  to refer to fields shadowed in the subclass (although shadowing fields is
  discouraged in Java).

\item
  \code{<self>}, i.e. the value of the annotated reference (non-primitive) variable.
  Currently only defined for the \<@GuardedBy> type system.
  For example, \code{@GuardedBy("<self>") Object o} indicates that the value
  referenced by \<o> is guarded by the intrinsic (monitor) lock of the value
  referenced by \<o>.

\item
  a formal parameter, represented as \<\#> followed by the \textbf{one-based} parameter
  index.  For example: \<\#1>, \<\#3>.  It is not permitted to write \<\#0> to
  refer to the receiver object; use \<this> instead.

  The formal parameter syntax \<\#1> is less natural in source code
  than writing the formal parameter name.  This syntax is necessary for
  separate compilation, when an annotated method has already been compiled
  into a \<.class> file and a client of that method is later compiled.
  In the \<.class> file, no formal parameter name information is available,
  so it is necessary to use a number to indicate a formal parameter.
  If all your annotated code is always provided to the checker in source
  code format (this also implies that you will never provide \<.class>
  files to clients who might want to do type-checking), then you can use
  the formal parameter name instead of the numeric index.
%% At some point, we will start assuming that the .class file was created
%% by the Checker Framework and this justification, and the #n syntax
%% itself, will not be necessary.
% Running
% the compiler without a checker should create legal annotations in the
% \<.class> file, so we cannot rely on the checker to translate names to
% indices.

\item
  a local variable.  Write the variable name.  For example: \<myLocalVar>.
  The variable must be in scope; for example, a method annotation on method
  \<m> cannot mention a local variable that is declared inside \<m>.

\item
  a static variable.  Write the class name and the variable, as in
  \<System.out>.

\item
  a field of any expression.  For example:  \<next>,
  \<this.next>, \<\#1.next>. %, \<myLocalVar.next>.
  You may optionally omit a leading ``\<this.>'', just as in Java.  Thus,
  \<this.next> and \<next> are equivalent.

\item
  an array access.  For example:  \<this.myArray[i]>, \<vals[\#1]>.

\item
  an array creation. For example: \<new int[10]>, \<new String[] {"a", "b"}>.

\item literals: string, integer, char, long, float, double, null, class literals.

\item a method invocation on any expression.
  This even works for overloaded methods and methods with type parameters.
  For example:
  \<m1(x, y.z, \#2)>, \<a.m2("hello")>.

  Currently, the Checker Framework cannot prove all contracts about method
  calls, so you may need to suppress some warnings.

  One unusual feature of the Checker Framework's Java expressions is that a
  method call is allowed to have side effects.  Other tools forbid methods
  with side effects (and doing so is necessary if a specification is going
  to be checked at run time via assertions).  The Checker Framework enables
  you to state more facts.  For example, consider the annotation on
  \<java.io.BufferedReader.ready()>:

\begin{Verbatim}
  @EnsuresNonNullIf(expression="readLine()", result=true)
  @Pure public boolean ready() throws IOException { ... }
\end{Verbatim}

  This states that if \<readLine()> is called immediately after \<ready()>
  returns true, then \<readLine()> returns a non-null value.

\end{itemize}

% We want this in the future:
% The expression may refer to private fields and method calls.  The client
% cannot directly make use of these private fields and method calls, but they
% may be useful in establishing a precondition that itself refers to private
% fields or methods.  This exposure of implementation details is unfortunate,
% but it enables more expressive and useful specifications to be written, it does not jeopardize soundness nor let the client manipulate the
% representation, and it is simple
% simpler than defining specification fields or ghost fields
%% TODO: we may eventually have an @SpecField annotation


\textbf{Limitations:}
It is not possible to write a
quantification over all array components (e.g., to express that all
  array elements are non-null).  There is no such Java expression, but it
  would be useful when writing specifications.


\sectionAndLabel{Field invariants}{field-invariants}

Sometimes a field declared in a superclass has a more precise type in a
subclass.  To express this fact, write
\refqualclass{framework/qual}{FieldInvariant} on the subclass. It specifies
the field's type in the class on which this annotation is written.
The field must be declared in a superclass and
must be final.

% Other possible examples:
% Vehicles have some number of wheels; motorcycles have exactly 2 wheels.
% Sentence: subject (missing for imperative, which would let us use
%   @Unused, but that's not the point here), verb, object (optional,
%   missing for some types of sentences?).
% Book might have an index
% DOM tree might have parent or not.
% Computer might have a sound card or not.

For example,
\begin{Verbatim}
class Person {
  final @Nullable String nickname;
  public Person(@Nullable String nickname) {
    this.nickname = nickname;
  }
}

// A rapper always has a nickname.
@FieldInvariant(qualifier = NonNull.class, field = "nickname")
class Rapper extends Person {
  public Rapper(String nickname) {
    super(nickname);
  }
  void method() {
    ... nickname.length() ...   // legal, nickname is non-null in this class.
  }
}
\end{Verbatim}
 A field invariant annotation can refer to more than one field. For example,
\<@FieldInvariant(qualifier = NonNull.class, field = \{fieldA, fieldB\})> means
that \<fieldA> and \<fieldB> are both non-null in the class upon which the
annotation is written.  A field invariant annotation
can also apply different qualifiers to different fields. For example,
\<@FieldInvariant(qualifier = \{NonNull.class, Untainted.class\}, field =
\{fieldA, fieldB\})> means that \<fieldA> is non-null and \<fieldB> is untainted.

This annotation is inherited:  if a superclass is annotated with
\<@FieldInvariant>, its subclasses have the same annotation. If a subclass has its
own \<@FieldInvariant>, then it must include the fields in the superclass
annotation and those fields' annotations must be a subtype (or equal) to the
annotations for those fields in the superclass \<@FieldInvariant>.

Currently, the \<@FieldInvariant> annotation is trusted rather than
checked.  In other words, the \<@FieldInvariant> annotation introduces a
loophole in the type system, which requires verification by other means
such as manual examination.


% \sectionAndLabel{Unused fields and dependent types}{unused-fields-and-dependent-types}
\sectionAndLabel{Unused fields}{unused-fields}

In an inheritance hierarchy, subclasses often introduce new methods and
fields.  For example, a \<Marsupial> (and its subclasses such as
\<Kangaroo>) might have a variable \<pouchSize> indicating the size of the animal's
pouch.  The field does not exist in superclasses such as
\<Mammal> and \<Animal>, so Java issues a compile-time
error if a program tries to access \<myMammal.pouchSize>.

If you cannot use subtypes in your program, you can enforce similar
requirements using type qualifiers.
For fields, use the \<@Unused> annotation (Section~\ref{unused-annotation}), which enforces that a field or method may only
be accessed from a receiver expression with a given annotation (or one of
its subtypes).
For methods, annotate the receiver parameter \<this>; then a method call
type-checks only if the actual receiver is of the specified type.

% For example, consider the declaration
%   void getChars(@Untainted String this, int srcBegin, int srcEnd, char[] dst,
%   int dstBegin) { ... }
% Now, the invocation
%   myString.getChars(a, b, c, d)
% type-checks only if myString has type @Untainted String.  It does not
% type-check if myString has type @Tainted String.


% Then,
% Section~\ref{dependent-types} describes an even more powerful mechanism, by
% which the qualifier of a field depends on the qualifier of the expression
% from which the field was accessed.

Also see the discussion of typestate checkers, in
Chapter~\ref{typestate-checker}.


% \subsectionAndLabel{Unused fields}{unused-fields}
\subsectionAndLabel{\<@Unused> annotation}{unused-annotation}

A Java subtype can have more fields than its supertype.  For example:

\begin{Verbatim}
class Animal {}
class Mammal extends Animal { ... }
class Marsupial extends Mammal {
  int pouchSize;  // pouch capacity, in cubic centimeters
  ...
}
\end{Verbatim}

You can simulate
the same effect for type qualifiers:
the \refqualclass{framework/qual}{Unused} annotation
on a field declares that the field may \emph{not} be accessed via a receiver of
the given qualified type (or any \emph{super}type).
% a given field \emph{may not} be accessed via
% a reference with a supertype qualifier, but \emph{may} be accessed via a reference
% with a subtype qualifier.
%% The following is true, but there's no need to distract readers with this detail.
% (It would probably be clearer to replace \<@Unused> by an annotation that
% indicates when the field \emph{may} be used.)
For example:

\begin{Verbatim}
class Animal {
  @Unused(when=Mammal.class)
  int pouchSize;  // pouch capacity, in cubic centimeters
  ...
}
@interface Mammal {}
@interface Marsupial {}

@Marsupial Animal joey = ...;
... joey.pouchSize ...    // OK
@Mammal Animal mae = ...;
... mae.pouchSize ...    // compile-time error
\end{Verbatim}

The above class declaration is like writing

\begin{Verbatim}
class @Mammal-Animal { ... }
class @Marsupial-Animal {
  int pouchSize;  // pouch capacity, in cubic centimeters
  ...
}
\end{Verbatim}


% \subsectionAndLabel{Dependent types}{dependent-types}
%
% A variable has a \emph{dependent type} if its type depends on some other
% value or type.
% %  --- the type is dynamically, not statically, determined.
% % (Type-safety can still be statically determined, though.)
%
% The Checker Framework supports a form of dependent types, via the
% \refqualclass{framework/qual}{Dependent} annotation.
% This annotation changes the type of a reference, based on the
% qualified type of the receiver (\code{this}).  This can be viewed as a more
% expressive form of polymorphism (see Section~\ref{polymorphism}).  It can
% also be seen as a way of linking the meanings of two type qualifier
% hierarchies.
%
% When the \refqualclass{framework/qual}{Unused} annotation is sufficient, you
% should use it instead of \code{@Dependent}.
%
% Here is a restatement of the example of Section~\ref{unused-fields}, using
% \refqualclass{framework/qual}{Dependent}:
%
% \begin{Verbatim}
% @interface Mammal {}
% @interface Marsupial {}
% class Animal {
%   // pouch capacity, in cubic centimeters
%   // (non-null if this animal is a marsupial)
%   @Nullable @Dependent(result=NonNull.class, when=Marsupial.class) Integer getPouchSize;
%   ...
% }
%
% @Marsupial Animal joey = ...;
% ... joey.getPouchSize().intValue() ...    // OK
% @Mammal Animal mae = ...;
% ... mae.getPouchSize().intValue() ...    // compile-time error:
%                                          //   dereference of possibly-null mae.getPouchSize()
% \end{Verbatim}
%
% Just as writing \<@Unused> is similar to writing multiple classes (but when
% it is not possible to write real subclasses), \<@Dependent> mimics the
% effect of multiple classes with overriding definitions of some method or
% field.
% The above class declaration is like writing
%
% \begin{Verbatim}
% class @Mammal-Animal {
%   // pouch capacity, in cubic centimeters
%   // (non-null if this animal is a marsupial)
%   @Nullable Integer getPouchSize();
% }
% class @Marsupial-Animal {
%   // pouch capacity, in cubic centimeters
%   @NonNull Integer getPouchSize();
%   ...
% }
% \end{Verbatim}
%
%
% \subsubsectionAndLabel{Limitations of \<@Dependent>}{dependent-types-limitations}
%
% It is unsound to write \<@Dependent> on a non-\<final> field.  Consider the
% following:
%
% \begin{Verbatim}
% class MyClass {
%   @Nullable @Dependent(result=NonNull.class, when=Marsupial.class) Integer pouchSize;
% }
% \end{Verbatim}
%
% Then it would be possible to write
%
% \begin{Verbatim}
% @Marsupial Animal a = new @Marsupial Animal();
% @Mammal Animal b = a;
% b.pouchSize = null;
% a.pouchSize.intValue();
% \end{Verbatim}
%
% \noindent
% In the last line of the example, \<a.pouchSize> is null, contradicting its
% declaration and leading to a null pointer exception that would not be
% caught by the type-checker.
%
% In certain circumstances, it may be desirable to write such an annotation
% on a \<private> field anyway, manually check all uses, and then suppress
% the warning.
%
% It is sound to write \<@Dependent> on a \<final> field.  Such a field
% behaves like a getter method, such as \<getPouchSize()> above.



% TO DO:  give an example where @Dependent is actually needed


% LocalWords:  MyClass qual PolymorphicQualifier DefaultQualifier subpackages
% LocalWords:  actuals toArray CollectionToArrayHeuristics nn RegexBottom
% LocalWords:  MyList Nullness DefaultLocation nullness PolyNull util java TODO
% LocalWords:  QualifierDefaults nullable lub persistData updatedAt nble KeyFor
% LocalWords:  subtype's RequiresNonNull EnsuresNonNull EnsuresNonNullIf
% LocalWords:  myLocalVar myClass getPackage getSuperclass myString Regex
% LocalWords:  getComponentType enum implementers dereferenced superclasses
%  LocalWords:  regex myStrings myVar pouchSize myMammal getter foo MyAnno
%  LocalWords:  getPouchSize TerminatesExecution myvar myField getField m1
%  LocalWords:  computeValue AsuggestPureMethods Instanceof arg myInt Anno
%  LocalWords:  AcheckPurityAnnotations AassumeSideEffectFree iMplicit instanceof m2
%  LocalWords:  AassumeAssertionsAreEnabled myArray vals propkey forName
%  LocalWords:  fenum i18n RequiresQualifier EnsuresQualifier ClassName
%  LocalWords:  EnsuresQualifierIf AsuppressWarnings AinvariantArrays asts
%  LocalWords:  formatter ocals nstanceof plicit ounds wildcard's da
%%  LocalWords:  wildcards guieffect AassumeAssertionsAreDisabled readLine
%%  LocalWords:  AssumeAssertion AsafeDefaultsForUnannotatedBytecode
%%  LocalWords:  I18nFormatFor AunsafeDefaultsForUnannotatedBytecode
%%  LocalWords:  AuseConservativeDefaultsForUncheckedCode DefaultFor TypeUseLocation
% LocalWords:  EnsuresKeyFor EnsuresKeyForIf
% LocalWords:  EnsuresLockHeld EnsuresLockHeldIf FieldInvariant fieldA
% LocalWords:  fieldB myMethod SomeType runtime typedefs signedness
%%  LocalWords:  Signedness MonotonicNonNull UpperBoundFor
% LocalWords:  getTypeDeclarationBounds identityHashCode
